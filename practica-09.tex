\section{Practica 9 - Planaridad}
\subsection{Definiciones generales}
\begin{enumerate}
\item Una representación planar de un grafo $G$ es un conjunto de puntos en el plano que se corresponden con los vertices de $G$ unidos por curvas que se corresponden con las aristas de $G$, sin que estas se crucen entre si.
\item Un grafo es planar si admite representación planar.
\item Una region de una rep. planar de $G$ es el conjunto de todos los puntos alcanzables desde un punto (que no sea un vertice ni parte de una arista) sin atravesar vertices ni aristas.
\item Toda representación planar de un grafo tiene exactamente una region de area infinita, la region exterior.
\item La frontera de una region es el circuito que rodea a la region (puede tener vertices y aristas repetidas)
\item El grado/tamaño de una region es el nro. de aristas que tiene su frontera.
\item \textit{Propiedad:} $K_{5}$ es el grafo no planar con menor cantidad de vertices y $K_{3,3}$ es el grafo no planar con menor cantidad de aristas.
\item \textit{Propiedad:} Si un grafo contiene un subgrafo no planar, es no planar.
\item \textit{Propiedad:} $G$ planar $\Rightarrow$ toda componente conexa de $G$ es planar y todo subgrafo es planar.
\end{enumerate}

\subsection{Subdivisión}
\begin{enumerate}
\item Subdividir una arista $e=(v,w)$ consiste en agregar $u \notin V$ a $G$ y reemplazar $e$ por dos aristas $e'=(v,u)$ y $e''=(u,w)$
\item Un grafo $G'$ es subdivisión de otro grafo $G$ si puede obtenerse a partir de sucesivas operaciones de subdivisión.
\item $G y G'$ son homeomorfos si hay un isomorfismo entre una subdivisión de $G$ y una de $G'$.
\item \textit{Propiedad:} Si $G'$ es una subdivisión de $G$, entonces $G$ es planar $\iff$ $G'$ es planar.
\item \textit{Propiedad:} La planaridad es invariante bajo homeomorfismo.
\item \textit{Corolario:} Si un grafo $G$ tiene un subgrafo homeomorfo a un grafo no planar entonces $G$ es no planar.
\item \textit{Teorema:} (Kuratowski) Un grafo es planar $\iff$ no contiene ningún subgrafo homeomorfo a $K_{5} o K_{3,3}$.
\end{enumerate}

\subsection{Contracción}
\begin{enumerate}
\item Contraer una arista $e=(v,w)$ consiste en eliminar $e$ del grafo y considerar sus extremos como un vertice $u \notin V$ quedando como aristas incidentes a $u$ las incidentes a $v$ y $w$.
\item $G'$ es contracción de $G$ si se puede obtener a partir de aplicar sucesivas operaciones de contracción sobre $G$. En este caso se dice que $G$ es contraible a $G'$.
\item \textit{Teorema:} (Whitney) $G$ es planar $\iff$ no contiene subgrafo contraible a $K_{5} o K_{3,3}$.
\end{enumerate}

\subsection{Otras propiedades}
\begin{enumerate}
\item \textit{Teorema:} (Euler) $G$ conexo planar, entonces una representación planar de $G$ tiene $r=m-n+2$ regiones en el plano.
\item \textit{Corolario:} $G$ simple, conexo, planar $n \geq 3 \Rightarrow m \leq 3n-6$
\item \textit{Corolario:} $G$ simple, conexo, bipartito, planar $n \geq 3 \Rightarrow m \leq 2n-4$
\item $G*$ es el grafo dual de $G$ planar, tiene un vertice por cada region de $G$ y una arista uniendo dos vertices correspondientes a dos regiones $r_{1}$ y $r_{2}$ por cada arista en la frontera $r_{1}$ y $r_{2}$ en $G$.
\item Cualquier arbol es planar.
\item \textit{Propiedad:} $G$ conexo, planar, simple $n \geq 3 \Rightarrow$ el tamaño de la frontera de cada region es mayor o igual a $3$.
\item El grafo dual de un grafo dual de $G$ es $G$ ($G$ conexo).
\item Los grafos 3-conexos tienen una unica representación planar.
\item Un grafo k-conexo cuando se le pueden sacar cualquier $k$ nodo y seguir siendo conexo.
\item Si $G$ es planar $\Rightarrow \sum_{\substack{f \in F}}|f| = 2m$ (los tamaños de todas las fronteras) Recordar $\sum_{\substack{v \in V}}d(v) = 2m$
\end{enumerate}

\subsection{Algoritmos}
\paragraph{Demoucron , Malgrange y Pertuiset}
Consideraciones previas:
\begin{itemize}
    \item Una \textit{parte p de G relativa a R} se define como:
        \begin{itemize}
            \item Un eje $(u,v) \in E(G \setminus R)$ con $u,\ v \in V(R)$ ó
            \item Una componente conexa de $G \setminus R$ junto con todos los ejes de $G$ incidentes a ella (ejes colgantes).
        \end{itemize}
    \item Una representación planar de un subgrafo $R$ de $G$ es \textit{G-admisible} si se puede extender dicha representación a una representación planar de $G$.
    \item Dada una parte $p$ de $R$ relativa a $G$, un nodo $v$ es \textit{nodo de contacto de p} sii $v \in V(R)$ y $v$ es extremo de algún eje de $p$.
    \item Dada una parte $p$ de $R$ relativa a $G$ y una representación $\tilde R$, decimos que es \textit{dibujable en una región f} de $\tilde R$ si existe una extensión de $\tilde R$ donde $p$ queda en $f$.
    \item Dada una parte $p$ de $R$ relativa a $G$ y una representación $\tilde R$, decimos que es \textit{potencialmente dibujable en una región f} de $\tilde R$ si todo nodo de contacto de $p$ pertenece a la frontera de $f$.
    \item \textit{Teorema:} Dada una parte $p$ de $R$ relativa a $G$ y una representación $\tilde R$, $p$ es dibujable en $f$ $\iff$ $p$ es potencialmente dibujable en $f$.
    \item Llamamos $F(p, \tilde R)$ al conjunto de regiones de $\tilde R$ donde $p$ es dibujable.
    \item \textit{Teorema:} Si $\tilde R$ es G-admisible, entonces toda parte $p$ de $R$ relativa a $G$ satisface $F(p, \tilde R) \neq \emptyset$.
\end{itemize}

Complejidad: $\mathcal{O}(n^2)$

Pseudocódigo:
\begin{algorithmic}[1]
    \State Buscar un ciclo $G_0$ y construir una representación planar de $G_0$
    \State $i \gets 0$
    \While{$E(G) \setminus E(G_i) \neq \emptyset$}
        \ForAll{parte $p$ de $R$ relativa a $G_i$}
            \State Calcular $F(p, \tilde G_ i)$
            \If {$F(p, \tilde G_ i) = \emptyset$}
                \State \Return False
            \EndIf
        \EndFor
        \If {$\exists\ p : F(p, \tilde G_ i) = \{f\}$}
            \State Elegir $f$ como el único $F(p, \tilde G_ i)$
        \Else
            \State Elegir cualquier $p$ y un $f \in F(p, \tilde G_ i)$ para el $p$ elegido
        \EndIf
        \State Buscar una ruta $P_i$ que conecte dos nodos de contacto de $p$
        \State $G_{i+1} \gets G_i \cup P_i$
        \State Dibujar $P_i$ en $f$ para obtener $\tilde  G_{i+1}$
        \State $i \gets i+1$
    \EndWhile
    \State \Return True // $\tilde G_i$ es una representación planar de G
\end{algorithmic}
