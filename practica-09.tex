\section{Practica 9 - Planaridad}
\subsection{Definiciones, Propiedades y Teoremas}
\begin{enumerate}
\item Una representacion planar de un grafo $G$ es un conjunto de puntos en el plano que se corresponden con los vertices de $G$ unidos por curvas que se corresponden con las aristas de $G$, sin que estas se crucen entre si.
\item Un grafo es planar si admite representacion planar.
\item Una region de una rep. planar de $G$ es el conjunto de todos los puntos alcanzables desde un punto (que no sea un vertic e ni parte de una arista) sin atravesar vertices ni aristas.
\item Toda representacion planar de un grafo tiene exactamente una region de area infinita, la region exterior.
\item La frontera de una region es el circuito que rodea a la region (puede tener vertices y aristas repetidas)
\item El grado/tamaño de una region es el nro. de aristas que tiene su frontera.
\item $K_{5}$ es el grafo no planar con menos $n$; $K_{3,3}$ es el no planar de menos $m$.
\item \textit{Propiedad:} Si un grafo contiene un subgrafo no planar, es no planar.
\item Subdividir una arista $e=(v,w)$ consiste en agregar $u \notin V$ a $G$ y reemplazar $e$ por dos aristas $e'=(v,u) y e''=(u,w)$
\item Un grafo $G'$ es subdivision de $G$ si puede obtenerse subdividiendo a $G$.
\item $G y G'$ son homeomorfos si hay isomorfismo entre alguna subdivision de $G$ con alguna de $G'$.
\item \textit{Teorema:} (Kuratowski) Un grafo es planar $\iff$ no contiene ningun subgrafo homeomorfo a $K_{5} o K_{3,3}$.
\item \textit{Propiedad:} Si un grafo G tiene un subgrafo homeomorfo a un grafo no planar entonces $G$ es no planar.
\item $G$ planar $\Rightarrow$ toda comp conexa de $G$ es planar y todo subgrafo es planar.
\item \textit{Propiedad:} La planaridad es invariante bajo homeomarfismo.
\item \textit{Propiedad:} $G'$ subdivision de $G$, entonces $G$ es planar $\iff G'$ es planar.
\item Contraer una arista $e=(v,w)$ consiste en eliminar $e$ del grafo y considerar sus extremos como un vertice $u \notin V$ quedando como aristas incidentes a $u$ las incidentes a $v$ y $w$.
\item $G'$ es contraccion de $G$ si se puede obtener haciendo contracciones de $G$.
\item \textit{Teorema:} (Whitney) $G$ es planar $\iff$ no contiene subgrafo contraible a $K_{5} o K_{3,3}$.
\item \textit{Teorema:} (Euler) $G$ conexo planar, una representacion planar de $G$ tiene $r=m-n+2$ regiones en el plano.
\item \textit{Corolario:} $G$ simple, conexo, planar $n \geq 3 \Rightarrow m \leq 3n-6$
\item \textit{Corolario:} $G$ simple, conexo, bipartito, planar $n \geq 3 \Rightarrow m \leq 2n-4$
\item $G*$ es el grafo dual de $G$ planar, tiene un vertice por cada region de $G$ y una arista uniendo dos vertices correspondientes a dos regiones $r_{1}$ y $r_{2}$ por cada arista en la frontera $r_{1}$ y $r_{2}$ en $G$.
\item Al subdividir una arista se agregar nodos de grado 2
\item Cualquier arbol es planar.
\item \textit{Propiedad:} $G$ conexo, planar, simple $n \geq 3 \Rightarrow$ el tamaño de la frontera de cada region es mayor o igual a $3$.
\item El grafo dual de un grafo dual de $G$ es $G$ ($G$ conexo).
\item los grafos 3-conexos tienen una unica representacion planar.
\item un grafo k-conexo cuando se le pueden sacar cualquier $k$ nodo y seguir siendo conexo.
\item si $G$ es planar $\Rightarrow \sum_{\substack{f \in F}}|f| = 2m$ (los tamaños de todas las fronteras) Recordar $\sum_{\substack{v \in V}}d(v) = 2m$
\end{enumerate}
