\documentclass[a4paper, 11pt]{article}
\usepackage{verbatim} 
\usepackage{amsmath}
\usepackage{amsfonts}
\usepackage{amssymb}
\usepackage{amsthm}
\usepackage{siunitx}
\sisetup{
    %output-decimal-marker={,}% just uncomment if you want to use comma as the decimal marker!
}
\usepackage{listings}
\usepackage[utf8]{inputenc}
\usepackage[spanish, activeacute]{babel}
\usepackage[usenames,dvipsnames]{color}
\usepackage[width=15.5cm, left=3cm, top=2.5cm, height= 24.5cm]{geometry}
\usepackage{graphicx}
%\usepackage{subcaption}
\usepackage[all]{xy}
\usepackage{multicol}
\usepackage{subfig}
\usepackage{algorithm}
\usepackage{algorithmic}
\usepackage{cancel}
\usepackage{array}
\usepackage{float}
\usepackage{xcolor}
\usepackage{color,hyperref}
\setcounter{secnumdepth}{3} %%agrego subsubsection
\usepackage[nottoc,notlot,notlof]{tocbibind}

\newtheorem{lema}{Lema}
\newtheorem{teorema}{Teorema}
\newtheorem{corolario}{Corolario}
\newtheorem*{correctitud}{Correctitud del algoritmo}
\newtheorem*{notacion}{Notación}

\lstset{basicstyle=\small\ttfamily, breaklines=true, breakatwhitespace=true}
\lstset{numbers=left, numberstyle=\scriptsize}
\lstset{
     literate=%
         {á}{{\'a}}1
         {í}{{\'i}}1
         {é}{{\'e}}1
         {ý}{{\'y}}1
         {ú}{{\'u}}1
         {ó}{{\'o}}1
         {ě}{{\v{e}}}1
         {š}{{\v{s}}}1
         {č}{{\v{c}}}1
         {ř}{{\v{r}}}1
         {ž}{{\v{z}}}1
         {ď}{{\v{d}}}1
         {ť}{{\v{t}}}1
         {ñ}{{\~n}}1                
         {ů}{{\r{u}}}1
         {Á}{{\'A}}1
         {Í}{{\'I}}1
         {É}{{\'E}}1
         {Ý}{{\'Y}}1
         {Ú}{{\'U}}1
         {Ó}{{\'O}}1
         {Ě}{{\v{E}}}1
         {Š}{{\v{S}}}1
         {Č}{{\v{C}}}1
         {Ř}{{\v{R}}}1
         {Ž}{{\v{Z}}}1
         {Ď}{{\v{D}}}1
         {Ť}{{\v{T}}}1
         {Ň}{{\v{N}}}1                
         {Ů}{{\r{U}}}1    
}


%%%%%%%%%%%%%% ALGUNAS MACROS %%%%%%%%%%%%%%
% For \url{SOME_URL}, links SOME_URL to the url SOME_URL
\providecommand*\url[1]{\href{#1}{#1}}

\setlength{\parskip}{10pt plus 1pt minus 1pt}
\usepackage{tikz}
\def\checkmark{\tikz\fill[scale=0.4](0,.35) -- (.25,0) -- (1,.7) -- (.25,.15) -- cycle;}

% Same as above, but pretty-prints SOME_URL in teletype fixed-width font
\renewcommand*\url[1]{\href{#1}{\texttt{#1}}}

% Comando para poner el simbolo de Reales
\newcommand{\real}{\hbox{\bf R}}

\providecommand*\code[1]{\texttt{#1}}

%uso: \ponerGrafico{file}{caption}{scale}{label}
\newcommand{\ponerGrafico}[4]
{\begin{figure}[H]
	\centering
	\subfloat{\includegraphics[scale=#3]{#1}}
	\caption{#2} \label{fig:#4}
\end{figure}
}

\renewcommand{\algorithmiccomment}[1]{\hfill #1}

%%%%%%%%%%%%%%%%%%%%%%%%%%%%%%%%%%%%%%%%%%%%

\title{Teoremas Prácticos - Algoritmos y Estructuras de Datos III}

%\include{templates}

\begin{document}
\pagestyle{myheadings}
\maketitle
%\markboth{Nombre materia}{Nombre TP}

\thispagestyle{empty}
\tableofcontents

%\setcounter{section}{-1}
\newpage
\section{Practica 8}
\subsection{Demostraciones}
Lo importante de las demostraciones es \textbf{(1)} Entender por que tiene que pasar lo que piden demostrar \textbf{(2)} Una vez visto eso tratar de explicarlo traduciendo la idea con lenguaje matematico (nada mas una traduccion que sirve de herramienta, lo importante es explicar claro lo visto en \textbf{(1)})

\subsection{Euler}
\begin{enumerate}
\item Un circuito Euleriano es un circuito que recorre cada arista una unica vez. Un multigrafo es euleriano si tiene un circuito euleriano.
\item \textit{Teorema:} Son equivalentes para un (multi)grafo conexo:
   \begin{enumerate}
   \item G es Euleriano.
   \item Todo vertice de G tiene grado par.
   \item Las aristas de G pueden particionarse en circuitos.
   \end{enumerate}
\item Un camino euleriano es un camino que recorre cada arista una unica vez.
\item Un digrafo es euleriano si tiene un circuito orientado que pasa por cada arco exactamente una vez.
\item \textit{Teorema:} Un (multi)grafo tiene camino euleriano si y solo si todos sus vertices tienen grado par salvo dos (principio y fin).
\item \textit{Teorema:} Un digrafo conexo es euleriano si y solo si para todo vertice $v$ de $G$ se cumple $d_{in}(v) = d_{out}(v)$
\end{enumerate}

\subsection{Hamilton}
\begin{enumerate}
\item Un camino/circuito hamiltoneano en un grafo $G$ es un camino/circuito que recorre cada vertice una sola vez.
\item Un grafo es hamiltoneano si tiene circuito hamiltoneano.
\item \textit{Teorema:} (Condicion necesaria) Sea $G$ un grafo conexo. Si existe $ W \subset V$ tal que $G \setminus W$ tiene $c$ componentes conexas con $c > |W|$ entonces $G$ no es hamiltoneano. ${|W| = n(W) = |V(W)| }$
\item \textit{Teorema:} (Dirac; Cond. suficiente) Sea $G$ un grafo con $n \geq 3$ y para todo $v \in V$ se cumple $d(v) \geq \frac{n}{2}$
\item \textit{Propiedad:} G grafo con recurrencia grafica $(d_{1},...,d_{n})$ con $d_{1} \leq ... \leq d_{n}$ y $n \geq 3$. Si no existe $k < \frac{n}{2}$ tal que $d_{k} \leq k$ y $d_{n-k} \leq n-k$, entonces G es hamiltoneno.
\item \textit{Teorema:} (Ore, Cond. suficiente) $G=(V,E)$, $|V| = n \geq 3$ y para todo par de vertices $v,w \in V$ no adyacentes se cumple $d(v)+d(w)\geq n \Rightarrow G$ es hamiltoneno.
\item Todo circuito hamiltoneano es simple por que no pasa dos veces por el mismo nodo.
\item Para demostrar que un grafo es Hamiltoneano o Euleriano se puede hacer un algoritmo que devuelva el circuito y explicar por que es correcto.
\item "Pigeon hide principle": if $n$ items are put into $m$ pigeonhides with $n>m$, then at least one pigeonhide must contain more than one item.
\end{enumerate}

\section{Practica 9}
\subsection{Planaridad}
\begin{enumerate}
\item Una representacion planar de un grafo $G$ es un conjunto de puntos en el plano que se corresponden con los vertices de $G$ unidos por curvas que se corresponden con las aristas de $G$, sin que estas se crucen entre si.
\item Un grafo es planar si admite representacion planar.
\item Una region de una rep. planar de $G$ es el conjunto de todos los puntos alcanzables desde un punto (que no sea un vertic e ni parte de una arista) sin atravesar vertices ni aristas.
\item Toda representacion planar de un grafo tiene exactamente una region de area infinita, la region exterior.
\item La frontera de una region es el circuito que rodea a la region (puede tener vertices y aristas repetidas)
\item El grado/tamaño de una region es el nro. de aristas que tiene su frontera.
\item $K_{5}$ es el grafo no planar con menos $n$; $K_{3,3}$ es el no planar de menos $m$.
\item \textit{Propiedad:} Si un grafo contiene un subgrafo no planar, es no planar.
\item Subdividir una arista $e=(v,w)$ consiste en agregar $u \notin V$ a $G$ y reemplazar $e$ por dos aristas $e'=(v,u) y e''=(u,w)$
\item Un grafo $G'$ es subdivision de $G$ si puede obtenerse subdividiendo a $G$.
\item $G y G'$ son homeomorfos si hay isomorfismo entre alguna subdivision de $G$ con alguna de $G'$.
\item \textit{Teorema:} (Kuratowski) Un grafo es planar $\iff$ no contiene ningun subgrafo homeomorfo a $K_{5} o K_{3,3}$.
\item \textit{Propiedad:} Si un grafo G tiene un subgrafo homeomorfo a un grafo no planar entonces $G$ es no planar.
\item $G$ planar $\Rightarrow$ toda comp conexa de $G$ es planar y todo subgrafo es planar.
\item \textit{Propiedad:} La planaridad es invariante bajo homeomarfismo.
\item \textit{Propiedad:} $G'$ subdivision de $G$, entonces $G$ es planar $\iff G'$ es planar.
\item Contraer una arista $e=(v,w)$ consiste en eliminar $e$ del grafo y considerar sus extremos como un vertice $u \notin V$ quedando como aristas incidentes a $u$ las incidentes a $v$ y $w$.
\item $G'$ es contraccion de $G$ si se puede obtener haciendo contracciones de $G$.
\item \textit{Teorema:} (Whitney) $G$ es planar $\iff$ no contiene subgrafo contraible a $K_{5} o K_{3,3}$.
\item \textit{Teorema:} (Euler) $G$ conexo planar, una representacion planar de $G$ tiene $r=m-n+2$ regiones en el plano.
\item \textit{Corolario:} $G$ simple, conexo, planar $n \geq 3 \Rightarrow m \leq 3n-6$
\item \textit{Corolario:} $G$ simple, conexo, bipartito, planar $n \geq 3 \Rightarrow m \leq 2n-4$
\item $G*$ es el grafo dual de $G$ planar, tiene un vertice por cada region de $G$ y una arista uniendo dos vertices correspondientes a dos regiones $r_{1}$ y $r_{2}$ por cada arista en la frontera $r_{1}$ y $r_{2}$ en $G$.
\item Al subdividir una arista se agregar nodos de grado 2
\item Cualquier arbol es planar.
\item \textit{Propiedad:} $G$ conexo, planar, simple $n \geq 3 \Rightarrow$ el tamaño de la frontera de cada region es mayor o igual a $3$.
\item El grafo dual de un grafo dual de $G$ es $G$ ($G$ conexo).
\item los grafos 3-conexos tienen una unica representacion planar.
\item un grafo k-conexo cuando se le pueden sacar cualquier $k$ nodo y seguir siendo conexo.
\item si $G$ es planar $\Rightarrow \sum_{\substack{f \in F}}|f| = 2m$ (los tamaños de todas las fronteras) Recordar $\sum_{\substack{v \in V}}d(v) = 2m$ 
\end{enumerate}

\section{Practica 10}
\subsection{Coloreo}
\begin{enumerate}
\item Un coloreo valido de los nodos de un grafo $G=(V,E)$ es una asignacion $f:V\rightarrow C$ tal que $f(u) \neq f(v)$ $\forall(u,v)\in E$. Un k-coloreo de $G$ es un coloreo que usa exactamente $k$ colores. Un grafo es k-coloreable si existe un k-coloreo de $G$, $\chi(G)=k$.
\item el numero cromatico de $G, \chi(G)$ es el menor numero de colores necesarios para colorear los nodos de $G$.
\item \textit{Proposicion:} $H$ subgrafo de $G \Rightarrow \chi(H) \leq \chi(G)$.
\item Una clique es un subgrafo completo maximal. El numero clique $\omega(G)$ es el numero de nodos de la clique maxima.
\item Para cualquier grafo $G, \chi(G) \geq \omega(G)$
\item $\chi(K_{n}) = n$
\item $G$ bipartito con $m>0$ entonces $\chi(G) = 2$.
\item $H_{2k}$ es un circuito simple par entonces $\chi(H_{2k}) = 2$.
\item $H_{2k+1}$ es un circuito simple par entonces $\chi(H_{2k+1}) = 3$.
\item $T$ es un arbol con $n>1$ entonces $\chi(G) = 2$.
\item Si $\Delta(G)$ es el grado maximo de $G$ entonces $\chi(G) \leq \Delta(G)+1$
\item \textit{Teorema:} (Brooks) $G$ conexo, no completo, no circuito simple entonces $\chi(G) \leq \Delta(G)$
\item Lema: en todo $\Delta(G)$-coloreo de $G-{v}$ los vecinos de $v$ en $G$ usan todos los colores y $d(v) = \Delta(G)$ $\forall v \in V$.
\item $N(v)={v_{1},...,v_{\Delta(G)}}$ es la vecindad de v.
\item $\Delta(G)$-coloreo de $G-{v}$ donde $v_{i}$ tiene color $i$. Para $i \neq j$, $H_{ij}$ es el subgrafo inducido por los nodos de $G-{v}$ pintados con $i$ o $j$ en ese $\Delta(G)$-coloreo.
\item Lema: $v_{i}$ y $v_{j}$ pertenecen a la misma comp. conexa de $H_{ij}$
\item Lema: $P_{ij}$ comp. conexa de $H_{ij}$ que contiene a $v_{i}$ y $v_{j} \Rightarrow P_{ij}$ camino en $H_{ij}$.
\item $P_{ij} \cap P_{jk} = {v_{i}}$ para colores $i,j,k$ distintos.
\item \textit{Teorema:} (Haken) $G$ grafo planar $\Rightarrow \chi(G) \leq 4$. 
\item \textit{Teorema:} (Heawood) $G$ grafo planar $\Rightarrow \chi(G) \leq 5$. 
\item Coloreo de grafos es NP-C.
\item Grafos de Mycielski: $M_{i}$ tiene $p$ nodos, $M_{i+1}$ tiene $2p+1$ nodos (los $v_{i}$ de $M_{i}$, $w_{i}$ nuevos uno por cada $v_{i}$ y un nodo $z$ y aristas: las de $M_{i}$ mas aristas entre $w_{i}$ y $z$ mas aristas de cada $w_{i}$ con su par de $v_{i}$) $\chi(M_{i})=i$, $\omega(M_{i})=2$
\item Si $G$ no es conexo, $G^{c}$ lo es.
\item si $G$ es q-regular entonces $G^{c}$ es $(n-1-q)$-regular.
\item Si $G$ es completo: $P_{G}(k)= k(k-1)(k-2)(...)(k-n+1)$
\item Si $G$ no es completo: $P_{G}(k)= P_{G+e}(k) + P_{G*e}(k)$
\item Un coloreo valido de las aristas de $G$ es una asignacion de colores a las aristas en la cual dos aristas que tienen un nodo en comun tienen distinto color asignado.
\item Indice cromatico $\chi'(G)$ es el numero minimo de colores de un coloreo valido de las aristas.
\item \textit{Teorema:} (Vizing) $\Delta(G) \leq \chi'(G) \leq \Delta(G)+1$
\end{enumerate}
\newpage

\section{Practica 11}
\subsection{Matching}
\begin{enumerate}
\item Un matching entre los nodos de $G$ es un conjunto $M \subseteq X$ de aristas de $G$ tal que para todo $v \in V$, $v$ incide a lo sumo a una arista $e \in M$
\item Un Conjunto independiente (CI) de nodos de $G$ es un conjunto de nodos $I \in V$ tal que para toda arista $e \in X$, e incide a lo sumo a un nodo $v \in I$.
\item Un recubrimiento de los nodos de $G$ (RN) es un conjunto $R_{m}$ de aristas tal que todo $v \in V$ es incidente al menos a una arista $e \in R_{m}$.
\item Lema: $S \subseteq V$ es un CI $\iff$ $V \setminus S$ es un RA.
\item Un nodo v se dice saturado por un matching M si hay una arista de M incidente a v.
\item Dado un Matching M en G, un camino alternado en G con respecto a M es un camino simple donde se alternan las aristas de $X \setminus M$ con aristas M.
\item Dado un Matching M en G, un camino de aumento en G respecto de M es un camino alternado entre nodos no saturados por M.
\item Lema: $M_{0}$ y $M_{1}$ dos matchings en G y sea $G'=(V,X')$ con $X'=M_{0} \setminus M_{1} \cup M_{1} \setminus M_{0}$, entonces las componntes conexas de $G'$ son: nodos aislados, circuitos simple con aristas alternadamente en $M_{0}$ y $M_{1}$ o camino simple con aristas alternadamente en $M_{0}$ y $M_{1}$.
\item \textit{Teorema:} M matching maximo de G $\iff$ no existe camino de aumento en G con respecto a M.
\item \textit{Teorema:} G sin nodos aislados, M matching maximo y $R_{m}$ recubrimiento minimo de nodos, entonces $|M|+|R_{m}|=n$
\item \textit{Teorema:} Si I es un CI maximo de G y $R_{n}$ un recubrimiento minimo de aristas de G $\Rightarrow$ $|I|+|R_{n}|=n$.
\item Sea M un matching maximo en G, entonces los nodos no saturados por M son U con $|U|=n-2|M|$. U es un CI por que M es maximo.
\end{enumerate}
\subsection{Flujo}
\begin{enumerate}
\item Una red $N=(V,X)$ es un grafo orientado conexo que tiene dos nodos distinguidos una fuente s, con grado de salida positivo, y un sumidero t, con grado de entrada positivo.
\item Un flujo factible en una red $N=(V,X)$ con funcion de capacidad $c:X(G) \rightarrow \mathbb{R}_{\geq 0}$ es una funcion $f:X(G) \rightarrow \mathbb{R}_{\geq 0}$ tal que $0 \leq f(e) \leq c(e)$ $\forall$ arco $e \in X(G)$.

$\sum_{\substack{e \in In(v)}}f(e) = \sum_{\substack{e \in Out(v)}}f(e)$ para todo nodo $v \in V \setminus {s,t}$ hay conservacion de flujo.
\item El valor de flujo es $F=\sum_{\substack{e \in In(t)}}f(e) - \sum_{\substack{e \in Out(t)}}f(e) = \sum_{\substack{e \in Out(s)}}f(e) - \sum_{\substack{e \in In(s)}}f(e)$
\item Un corte en la red $N=(V,X)$ es un subconjunto $S\subseteq V$ tal que $s \in S$ y $t \notin S$
\item Dados $S,T \subseteq V$, conjunto $ST={(u\rightarrow v) \in X : u\in S y v\in T}$
\item f flujo en $N=(V,X)$ y S un corte de N $\Rightarrow$ $F=\sum_{\substack{e \in SS^{c}}}f(e) - \sum_{\substack{e \in S^{c}S}}f(e)$
\item capacidad de un corte S es $c(S)=\sum_{\substack{e \in SS^{c}}}c(e)$
\item f flujo con valor F y S corte de N $\Rightarrow$ $F \leq c(S)$
\item Corolario: F valor de un flujo de f de N y S un corte de N tal que $F=c(S) \Rightarrow$ f define dlujo maximo y S es corte de capacidad minima.
\item $N=(V,X)$ con funcion de capacidad c y flujo factible f definimos la red residual $R(N,f)=(V,X_{R})$ con $\forall (v\rightarrow w) \in X$,

$\exists (v\rightarrow w)\in X_{R}$ si $f(v\rightarrow w)< c(v\rightarrow w)$
$\exists (w\rightarrow v)\in X_{R}$ si $f(v\rightarrow w)> 0$
\item Un camino de aumento es un camino orientado P de s a t en $R(N,f)$.
\item \textit{Teorema:} f flujo definido sobre N, entonces f es un flujo maximo $\iff$ no ha camino de aumento en $R(N,f)$.
\item \textit{Teorema:} En una red N el valor del flujo maximo es igual a la capacidad del corte minimo.
\item Flujo maximo es P.
\item Complejidad Ford-Fulkerson $O(mnB)$ B cota superior para capacidades. (devuelve flujo maximo y corte minimo).
\item Complejidad Edmond y Karp $O(m^{2}n)$ (BFS de camino de Aumento para marcar nodos) 
\end{enumerate}

\section{Practica 12}
\subsection{P-NP}
\begin{enumerate}
\item Un algoritmo eficiente es un algoritmo de complejidad polinomial. Un problema esta bien resuelto si se conocen algoritmos eficientes para resolverlo.
\item Se estudian problemas de decision cuya respuesta es "si" o "no". Se puede extender a problemas de minimizacion (existe solucion menor que $k$?) o de maximizacion (solucion mayor que $k$).
\item Un problema es intratable si no puede ser resuelto por un algoritmo eficiente (por que es indecidible; por que la respuesta es exponencial; por que no se conocen mejores algoritmos.)
\item Un problema esta en P si existe una MTD de complejidad polinomial que lo resuelve (para toda instancia termina y contesta bien "si" o "no")
\item Un problema esta en NP si existe MTND polinomial que lo resuelve (toda instancia del problema con respuesta "si" tiene alguna rama que tiene una respuesta "si" y las instancias de "no" ninguna rama llega a "si")
\item Un problema esta en NP si dada una instancia de "si" y un certificado de la misma, puede ser verificada en tiempo polinomial. Certificado de tamaño polinomial.
\item Reduccion polinomial: $\Pi_{1}$ y $\Pi_{2}$ dos problemas de decision. Decimos que $f:$ Instancias$(\Pi_{2}) \rightarrow$ Instancias$(\Pi_{1})$ es reduccion polinomial de $\Pi_{2}$ en $\Pi_{1}$ si $f$ es una funcion polinomial y toda instancia $I_{2}$ de $\Pi_{2}$ con respuesta "si" cumple $f(I_{2})$ tiene respuesta "si" en $\Pi_{1}$. Escribimos $\Pi_{2} \leq_{p} \Pi_{1}$.
\item la composicion de dos reducciones polinomiales es una reduccion polinomial $\Rightarrow$ $\Pi_{1} \leq_{p} \Pi_{2}$ y $\Pi_{2} \leq_{p} \Pi_{3}$ $\Rightarrow$ $\Pi_{1} \leq_{p} \Pi_{3}$
\item Un problema $\Pi$ esta en NP-Completo si: $\Pi \in$ NP, y para todo $\Pi_{i} \in$ NP, $\Pi_{i} \leq_{p} \Pi$
\item Un problema esta en NP-Hard si para todo $\Pi_{i} \in$ NP, $\Pi_{i} \leq_{p} \Pi$.
\item Si existe un problema $\Pi \in$ (NP-C $\cap$ P) entonces $P=NP$
\item Si existe un problema $\Pi \in$ (NP $\setminus$ P) entonces $P\neq NP$
\item \textit{Teorema:} (Cook 1971) SAT es NP-Completo.
\item Para probar que un problema $\Pi_{2}$ es NP-Completo hay que:
   \begin{enumerate}
   \item $\Pi_{2} \in$ NP (dar algoritmo y estructuras)
   \item Elegir problema $\Pi_{1} \in$ NP-Completo y construir una reduccion polinomial $\Pi_{1} \leq_{p} \Pi_{2}$, existe $f$ tal que
      \begin{enumerate}
      \item $f$ es polinomial
      \item Transforma instancias de $\Pi_{1}$ a $\Pi_{2}$ (dar algoritmo y estructuras)
      \item Para toda instancias $I_{1}$ de $\Pi_{1}$ y resp($I_{1}$)="si" $\Rightarrow$ resp($f(I_{1})$)="si" (explicar grafos)
      \item Para toda instancias $I_{1}$ de $\Pi_{1}$ y resp($f(I_{1})$)="si" $\Rightarrow$ resp($I_{1}$)="si"
      \end{enumerate}
   \end{enumerate}
\end{enumerate}

\end{document}
