\section{Practica 10 - Coloreo}
\subsection{Definiciones, Propiedades y Teoremas}
\begin{enumerate}
\item Un coloreo válido de los nodos de un grafo $G=(V,E)$ es una asignacion $f:V\rightarrow C$ tal que $f(u) \neq f(v)$ $\forall(u,v)\in E$. Un k-coloreo de $G$ es un coloreo que usa exactamente $k$ colores. Un grafo es k-coloreable si existe un k-coloreo de $G$.
\item El número cromático de $G, \chi(G)$, es el menor número de colores necesarios para colorear los nodos de $G$. Se dice que $G$ es k-cromático si $\chi(G) = k$. Ejemplos:
    \begin{itemize}
        \item $\chi(K_{n}) = n$
        \item $G$ bipartito con $m>0$ entonces $\chi(G) = 2$.
        \item $H_{2k}$ es un circuito simple par entonces $\chi(H_{2k}) = 2$.
        \item $H_{2k+1}$ es un circuito simple par entonces $\chi(H_{2k+1}) = 3$.
        \item $T$ es un arbol con $n>1$ entonces $\chi(G) = 2$.
    \end{itemize}
\item \textit{Propiedad:} $H$ subgrafo de $G \Rightarrow \chi(H) \leq \chi(G)$.
\item Una clique es un subgrafo completo máximal. El número clique $\omega(G)$ es el número de nodos de la clique máxima.
\item \textit{Propiedad:} Para cualquier grafo $G, \chi(G) \geq \omega(G)$
\item Grafos de Mycielski: $M_{i}$ tiene $p$ nodos, $M_{i+1}$ tiene $2p+1$ nodos (los $v_{i}$ de $M_{i}$, $w_{i}$ nuevos uno por cada $v_{i}$ y un nodo $z$ y aristas: las de $M_{i}$ mas aristas entre $w_{i}$ y $z$ mas aristas de cada $w_{i}$ con su par de $v_{i}$) $\chi(M_{i})=i$, $\omega(M_{i})=2$
\item \textit{Propiedad:} Si $\Delta(G)$ es el grado máximo de $G$ entonces $\chi(G) \leq \Delta(G)+1$
\item \textit{Teorema:} (Brooks) Sea $G$ un grafo conexo que no es un circuito impar ni un grafo completo, entonces $\chi(G) \leq \Delta(G)$
\item \textit{Lema:} en todo $\Delta(G)$-coloreo de $G-{v}$ los vecinos de $v$ en $G$ usan todos los colores y $d(v) = \Delta(G)$ $\forall v \in V$.
\item \textit{Teorema:} (Haken) $G$ grafo planar $\Rightarrow \chi(G) \leq 4$.
\item \textit{Teorema:} (Heawood) $G$ grafo planar $\Rightarrow \chi(G) \leq 5$.
\item $N(v)={v_{1},...,v_{\Delta(G)}}$ es la vecindad de v.
\item $\Delta(G)$-coloreo de $G-{v}$ donde $v_{i}$ tiene color $i$. Para $i \neq j$, $H_{ij}$ es el subgrafo inducido por los nodos de $G-{v}$ pintados con $i$ o $j$ en ese $\Delta(G)$-coloreo.
\item \textit{Lema:} $v_{i}$ y $v_{j}$ pertenecen a la misma comp. conexa de $H_{ij}$
\item \textit{Lema:} $P_{ij}$ comp. conexa de $H_{ij}$ que contiene a $v_{i}$ y $v_{j} \Rightarrow P_{ij}$ camino en $H_{ij}$.
\item $P_{ij} \cap P_{jk} = {v_{i}}$ para colores $i,j,k$ distintos.
\item Coloreo de grafos es NP-C.
% \item Si $G$ no es conexo, $G^{c}$ lo es.
% \item si $G$ es q-regular entonces $G^{c}$ es $(n-1-q)$-regular.
\item Si $G$ es completo: $P_{G}(k)= k(k-1)(k-2)(...)(k-n+1)$
\item Si $G$ no es completo: $P_{G}(k)= P_{G+e}(k) + P_{G*e}(k)$
\item Un coloreo válido de las aristas de $G$ es una asignacion de colores a las aristas en la cual dos aristas que tienen un nodo en comun tienen distinto color asignado.
\item Indice cromático $\chi'(G)$ es el número mínimo de colores de un coloreo válido de las aristas.
\item \textit{Teorema:} (Vizing) $\Delta(G) \leq \chi'(G) \leq \Delta(G)+1$
\end{enumerate}

\subsection{Algoritmos}
\paragraph{Algoritmo de coloreo secuencial}

Pseudocódigo:
\begin{algorithmic}[1]
    \State Entrada: Un grafo $G$ con un orden en los nodos $v_1, \dots, v_n$
    \State $f(v_1) \gets 1$
    \For{$i = 2,3,\dots,n$}
        \State $f(v_1) \gets min\{h\ /\ h \geq 1 \land f(v_j) \neq h\ \forall(v_j,v_i) \in E(G),\ 1 \leq j \leq i-1 \}$
    \EndFor
    \State \Return Coloreo definido por f
\end{algorithmic}
