\section{Practica 12 - Problemas NP}

\begin{enumerate}
\item Un algoritmo eficiente es un algoritmo de complejidad polinomial. Un problema esta bien resuelto si se conocen algoritmos eficientes para resolverlo.
\item Se estudian problemas de decision cuya respuesta es ``si'' o ``no''. Se puede extender a problemas de mínimizacion (existe solucion menor que $k$?) o de máximizacion (solucion mayor que $k$).
\item Un problema es intratable si no puede ser resuelto por un algoritmo eficiente (por que es indecidible; por que la respuesta es exponencial; por que no se conocen mejores algoritmos.)
\item Un problema esta en P si existe una Máquina de Turing Deterministica (MTD) de complejidad polinomial que lo resuelve (para toda instancia termina y contesta bien ``si'' o ``no'')
\item Un problema esta en NP si existe Máquina de Turing No Deterministica (MTND) polinomial que lo resuelve (toda instancia del problema con respuesta ``si'' tiene alguna rama que tiene una respuesta ``si'' y las instancias de ``no'' ninguna rama llega a ``si'')
\item Un problema esta en NP si dada una instancia de ``si'' y un certificado de la misma, puede ser verificada en tiempo polinomial. Certificado de tamaño polinomial.
\item Reduccion polinomial: $\Pi_{1}$ y $\Pi_{2}$ dos problemas de decision. Decimos que $f:$ Instancias$(\Pi_{2}) \rightarrow$ Instancias$(\Pi_{1})$ es reduccion polinomial de $\Pi_{2}$ en $\Pi_{1}$ si $f$ es una funcion polinomial y toda instancia $I_{2}$ de $\Pi_{2}$ con respuesta ``si'' cumple $f(I_{2})$ tiene respuesta ``si'' en $\Pi_{1}$. Escribimos $\Pi_{2} \leq_{p} \Pi_{1}$.
\item Las reducciones polinomilaes respetan transividad: $\Pi_{1} \leq_{p} \Pi_{2}$ y $\Pi_{2} \leq_{p} \Pi_{3}$ $\Rightarrow$ $\Pi_{1} \leq_{p} \Pi_{3}$
\item Un problema $\Pi$ esta en NP-Completo si: $\Pi \in$ NP, y para todo $\Pi_{i} \in$ NP, $\Pi_{i} \leq_{p} \Pi$
\item Un problema esta en NP-Hard si para todo $\Pi_{i} \in$ NP, $\Pi_{i} \leq_{p} \Pi$.
\item Si existe un problema $\Pi \in$ (NP-C $\cap$ P) entonces $P=NP$
\item Si existe un problema $\Pi \in$ (NP $\setminus$ P) entonces $P\neq NP$
\item \textit{Teorema:} (Cook 1971) SAT es NP-Completo.
\item Para probar que un problema $\Pi_{2}$ es NP-Completo hay que:
   \begin{enumerate}
   \item $\Pi_{2} \in$ NP (dar algoritmo y estructuras)
   \item Elegir problema $\Pi_{1} \in$ NP-Completo y construir una reduccion polinomial $\Pi_{1} \leq_{p} \Pi_{2}$, existe $f$ tal que
      \begin{enumerate}
      \item $f$ es polinomial
      \item Transforma instancias de $\Pi_{1}$ a $\Pi_{2}$ (dar algoritmo y estructuras)
      \item Para toda instancias $I_{1}$ de $\Pi_{1}$ y resp($I_{1}$)=``si'' $\Rightarrow$ resp($f(I_{1})$)=``si'' (explicar grafos)
      \item Para toda instancias $I_{1}$ de $\Pi_{1}$ y resp($f(I_{1})$)=``si'' $\Rightarrow$ resp($I_{1}$)=``si''
      \end{enumerate}
   \end{enumerate}
\end{enumerate}
